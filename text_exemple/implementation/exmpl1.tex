

\subsection{Description de l'implémentation}

Prenons $\Ddot{x}=-x$ comme oscillateur harmonique modèle, et qui sera écrit comme deux équations différentielles du premier ordre
  \begin{equation}
    \Dot{x}=p
  \end{equation}

\begin{equation}
    \Dot{p}=-x
\end{equation}

Nous allons intégrer numériquement ces équations par la méthode d'Euler avec conditions initiales ${\left[ 0,1 \right]}^2$ et un pas de temps $h=0,02$ 



La formule pour avancer dans le temps les équations différentielles (41) et (42) sont:
\begin{equation}
   x_{k+1}= x_{k} + hp_{k}   \;\;\;\;\; \;\;   p_{k+1}= p_{k} - hx_{k} 
\end{equation}

On peut écrire la formule utilisée dans ce document pour avancer d'un pas dans la notation matricielle comme
 \begin{equation}
   \left(
  \begin{array}{rrrrrrr}
     x_{k+1}    \\ 
       P_{k+1} \\ 
    
    \end{array}
    \right) =\left(
    \begin{array}{rrrrrrr}
      c & d    \\ 
      -d & c   \\ 
    
      \end{array}
       \right) \;\left(
    \begin{array}{rrrrrrr}
      x_{k}   \\ 
      p_{k}   \\ 
    
      \end{array}
       \right)
\end{equation}

où c et d sont des constantes différentes pour les différents schémas numériques, pour $h \rightarrow{0}$,  $c \rightarrow{1}$ et $d \rightarrow{h}$



\subsection{Discussion}
  Dans cet exemple deux propriétés sont intéressantes à vérifier:
  \begin{itemize}
      \itemperso{} les valeurs de sorties sont inférieures à 1?
      \itemperso{} sont elle bornées ?
  \end{itemize}
  
  
  
  
\subsection{Résultats}


\begin{itemize}
    \itemperso{Recherche de la valeur $Max$ de $x_{k}$} 
    
    prenons $Q= \begin{pmatrix} 
  1 & 0 \\
   0 & 0 
\end{pmatrix}$ et pour un vecteur 
 $\begin{pmatrix} 
  1 \\
   1  
\end{pmatrix}$, la valeur de $K$ sera égal à 221, $K_{1}$ sera égal à 113 
        
   Le $ Max = 1,219 $ à $k=48$, de ce fait la propriété $x_{k} \leq 1$ n'est pas vérifiée.
    
    
     \itemperso{Recherche de la valeur $Max$ de $v_{k}$}
     
       prenons $Q= \begin{pmatrix} 
  0 & 0 \\
   0 & 1 
\end{pmatrix}$ et pour un vecteur 
 $\begin{pmatrix} 
  1 \\
   1  
\end{pmatrix}$, la valeur de $K$ sera égal à 221, $K_{1}$ sera égal à $185 $ 
        
   Le $ Max = 1 $ à $k=0$.
     
     
      \itemperso{Pour la borne}
      
       prenons $Q= \begin{pmatrix} 
  1 & 0 \\
   0 & 1 
\end{pmatrix}$ et pour un vecteur 
 $\begin{pmatrix} 
  1 \\
   1  
\end{pmatrix}$, la valeur de $K$ sera égal à 132, $K_{1}$ sera égal à 132  
        
   Le $ Max = 1,79 $ à $k=0$.
      
\end{itemize}










