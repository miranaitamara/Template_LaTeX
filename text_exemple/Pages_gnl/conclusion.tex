\nnsection{Conclusion générale}

\subParagraphe{Conclusion}
\vskip .4cm

En conclusion, ce travail nous a permis d'avoir une première approche sur les problèmes d'optimisation des systèmes à temps discret.Le code développé propose un certain nombre de fonctionnalités de base utiles mais des points restent encore à étudier.


Ainsi, sur cette version finale, nous avons réussi à obtenir une solution précise avec un nombre fini d'évaluations des problèmes d'optimisations rencontrés, aussi Réussir à calculer les critères d’arrêt global pour les systèmes linéaires stables et les propriétés ellipsoïdales. 

Pour continuer ce projet, il serait aussi intéressant de mener un travail sur plusieurs fronts.Le point important focalise sur une grande classe de systèmes non linéaires qui est les systèmes non linéaires affines d'un côté, et d'essayer d'optimiser les entiers après avoir résoudre un problème de minimisation de l'autre côté.

\subParagraphe{Apport personnel}
\vskip .4cm

Ce stage a été très enrichissant pour moi car il m'a permis de découvrir dans le détail le secteur du mathématique basant sur la vérification de systèmes dynamiques tout en mettant l'accent sur l'optimisation mathématique.Aussi il m'a permis de découvrir concrètement la théorie de Lyapunov, la théorie des matrices ainsi la méthode prometteuse de l'optimisation conique qui est la SDP que j'ai particulièrement apprécié.

Ce stage m'a aussi permis de remémorer la programmation C qui est toujours d'actualité dans la programmation système et la robotique et de découvrir d'autres ressources, aussi d'autre librairies en langage C/C++ après l'avoir manipuler. 

Forte de cette expérience et en réponse à ses enjeux, j'aimerai beaucoup par la suite essayer de m'orienter via un autre projet  stage où j'aurais l'opportunité d'appliquer le peu que j'ai, vers le secteur des systèmes distribués et technologies des réseaux  avec des acteurs de petites tailles, et un important développement d'avenir. 




